%% bare_jrnl_compsoc.tex
%% V1.3
%% 2007/01/11
%% by Michael Shell
%% See:
%% http://www.michaelshell.org/
%% for current contact information.
%%
%% This is a skeleton file demonstrating the use of IEEEtran.cls
%% (requires IEEEtran.cls version 1.7 or later) with an IEEE Computer
%% Society journal paper.
%%
%% Support sites:
%% http://www.michaelshell.org/tex/ieeetran/
%% http://www.ctan.org/tex-archive/macros/latex/contrib/IEEEtran/
%% and
%% http://www.ieee.org/

%%*************************************************************************
%% Legal Notice:
%% This code is offered as-is without any warranty either expressed or
%% implied; without even the implied warranty of MERCHANTABILITY or
%% FITNESS FOR A PARTICULAR PURPOSE! 
%% User assumes all risk.
%% In no event shall IEEE or any contributor to this code be liable for
%% any damages or losses, including, but not limited to, incidental,
%% consequential, or any other damages, resulting from the use or misuse
%% of any information contained here.
%%
%% All comments are the opinions of their respective authors and are not
%% necessarily endorsed by the IEEE.
%%
%% This work is distributed under the LaTeX Project Public License (LPPL)
%% ( http://www.latex-project.org/ ) version 1.3, and may be freely used,
%% distributed and modified. A copy of the LPPL, version 1.3, is included
%% in the base LaTeX documentation of all distributions of LaTeX released
%% 2003/12/01 or later.
%% Retain all contribution notices and credits.
%% ** Modified files should be clearly indicated as such, including  **
%% ** renaming them and changing author support contact information. **
%%
%% File list of work: IEEEtran.cls, IEEEtran_HOWTO.pdf, bare_adv.tex,
%%                    bare_conf.tex, bare_jrnl.tex, bare_jrnl_compsoc.tex
%%*************************************************************************

% *** Authors should verify (and, if needed, correct) their LaTeX system  ***
% *** with the testflow diagnostic prior to trusting their LaTeX platform ***
% *** with production work. IEEE's font choices can trigger bugs that do  ***
% *** not appear when using other class files.                            ***
% The testflow support page is at:
% http://www.michaelshell.org/tex/testflow/




% Note that the a4paper option is mainly intended so that authors in
% countries using A4 can easily print to A4 and see how their papers will
% look in print - the typesetting of the document will not typically be
% affected with changes in paper size (but the bottom and side margins will).
% Use the testflow package mentioned above to verify correct handling of
% both paper sizes by the user's LaTeX system.
%
% Also note that the "draftcls" or "draftclsnofoot", not "draft", option
% should be used if it is desired that the figures are to be displayed in
% draft mode.
%
% The Computer Society usually requires 12pt for submissions.
%
\documentclass[10pt,journal,compsoc]{IEEEtran}
%
% If IEEEtran.cls has not been installed into the LaTeX system files,
% manually specify the path to it like:
% \documentclass[12pt,journal,compsoc]{../sty/IEEEtran}





% Some very useful LaTeX packages include:
% (uncomment the ones you want to load)


% *** MISC UTILITY PACKAGES ***
%
%\usepackage{ifpdf}
% Heiko Oberdiek's ifpdf.sty is very useful if you need conditional
% compilation based on whether the output is pdf or dvi.
% usage:
% \ifpdf
%   % pdf code
% \else
%   % dvi code
% \fi
% The latest version of ifpdf.sty can be obtained from:
% http://www.ctan.org/tex-archive/macros/latex/contrib/oberdiek/
% Also, note that IEEEtran.cls V1.7 and later provides a builtin
% \ifCLASSINFOpdf conditional that works the same way.
% When switching from latex to pdflatex and vice-versa, the compiler may
% have to be run twice to clear warning/error messages.






% *** CITATION PACKAGES ***
%
\ifCLASSOPTIONcompsoc
  % IEEE Computer Society needs nocompress option
  % requires cite.sty v4.0 or later (November 2003)
  % \usepackage[nocompress]{cite}
\else
  % normal IEEE
  % \usepackage{cite}
\fi
% cite.sty was written by Donald Arseneau
% V1.6 and later of IEEEtran pre-defines the format of the cite.sty package
% \cite{} output to follow that of IEEE. Loading the cite package will
% result in citation numbers being automatically sorted and properly
% "compressed/ranged". e.g., [1], [9], [2], [7], [5], [6] without using
% cite.sty will become [1], [2], [5]--[7], [9] using cite.sty. cite.sty's
% \cite will automatically add leading space, if needed. Use cite.sty's
% noadjust option (cite.sty V3.8 and later) if you want to turn this off.
% cite.sty is already installed on most LaTeX systems. Be sure and use
% version 4.0 (2003-05-27) and later if using hyperref.sty. cite.sty does
% not currently provide for hyperlinked citations.
% The latest version can be obtained at:
% http://www.ctan.org/tex-archive/macros/latex/contrib/cite/
% The documentation is contained in the cite.sty file itself.
%
% Note that some packages require special options to format as the Computer
% Society requires. In particular, Computer Society  papers do not use
% compressed citation ranges as is done in typical IEEE papers
% (e.g., [1]-[4]). Instead, they list every citation separately in order
% (e.g., [1], [2], [3], [4]). To get the latter we need to load the cite
% package with the nocompress option which is supported by cite.sty v4.0
% and later. Note also the use of a CLASSOPTION conditional provided by
% IEEEtran.cls V1.7 and later.


\usepackage{amssymb}

\usepackage{grffile}
 \usepackage[pdftex]{graphicx}

\usepackage{float}
\usepackage{perpage}
\MakeSorted{figure}
\MakeSorted{table}


% *** GRAPHICS RELATED PACKAGES ***
%
\ifCLASSINFOpdf

  
  % declare the path(s) where your graphic files are
  % \graphicspath{{/graficos/}{.}}
  % and their extensions so you won't have to specify these with
  % every instance of \includegraphics
%  \DeclareGraphicsExtensions{.pdf,.jpeg,.png}
\else
  % or other class option (dvipsone, dvipdf, if not using dvips). graphicx
  % will default to the driver specified in the system graphics.cfg if no
  % driver is specified.
   %\usepackage[dvips]{graphicx}
  % declare the path(s) where your graphic files are
 %  \graphicspath{{/graficos}}
  % and their extensions so you won't have to specify these with
  % every instance of \includegraphics
  % \DeclareGraphicsExtensions{.eps}
\fi
% graphicx was written by David Carlisle and Sebastian Rahtz. It is
% required if you want graphics, photos, etc. graphicx.sty is already
% installed on most LaTeX systems. The latest version and documentation can
% be obtained at: 
% http://www.ctan.org/tex-archive/macros/latex/required/graphics/
% Another good source of documentation is "Using Imported Graphics in
% LaTeX2e" by Keith Reckdahl which can be found as epslatex.ps or
% epslatex.pdf at: http://www.ctan.org/tex-archive/info/
%
% latex, and pdflatex in dvi mode, support graphics in encapsulated
% postscript (.eps) format. pdflatex in pdf mode supports graphics
% in .pdf, .jpeg, .png and .mps (metapost) formats. Users should ensure
% that all non-photo figures use a vector format (.eps, .pdf, .mps) and
% not a bitmapped formats (.jpeg, .png). IEEE frowns on bitmapped formats
% which can result in "jaggedy"/blurry rendering of lines and letters as
% well as large increases in file sizes.
%
% You can find documentation about the pdfTeX application at:
% http://www.tug.org/applications/pdftex





% *** MATH PACKAGES ***
%
\usepackage[cmex10]{amsmath}
% A popular package from the American Mathematical Society that provides
% many useful and powerful commands for dealing with mathematics. If using
% it, be sure to load this package with the cmex10 option to ensure that
% only type 1 fonts will utilized at all point sizes. Without this option,
% it is possible that some math symbols, particularly those within
% footnotes, will be rendered in bitmap form which will result in a
% document that can not be IEEE Xplore compliant!
%
% Also, note that the amsmath package sets \interdisplaylinepenalty to 10000
% thus preventing page breaks from occurring within multiline equations. Use:
%\interdisplaylinepenalty=2500
% after loading amsmath to restore such page breaks as IEEEtran.cls normally
% does. amsmath.sty is already installed on most LaTeX systems. The latest
% version and documentation can be obtained at:
% http://www.ctan.org/tex-archive/macros/latex/required/amslatex/math/

\usepackage{morefloats}



% *** SPECIALIZED LIST PACKAGES ***
%
%\usepackage{algorithmic}
% algorithmic.sty was written by Peter Williams and Rogerio Brito.
% This package provides an algorithmic environment fo describing algorithms.
% You can use the algorithmic environment in-text or within a figure
% environment to provide for a floating algorithm. Do NOT use the algorithm
% floating environment provided by algorithm.sty (by the same authors) or
% algorithm2e.sty (by Christophe Fiorio) as IEEE does not use dedicated
% algorithm float types and packages that provide these will not provide
% correct IEEE style captions. The latest version and documentation of
% algorithmic.sty can be obtained at:
% http://www.ctan.org/tex-archive/macros/latex/contrib/algorithms/
% There is also a support site at:
% http://algorithms.berlios.de/index.html
% Also of interest may be the (relatively newer and more customizable)
% algorithmicx.sty package by Szasz Janos:
% http://www.ctan.org/tex-archive/macros/latex/contrib/algorithmicx/




% *** ALIGNMENT PACKAGES ***
%
%\usepackage{array}
% Frank Mittelbach's and David Carlisle's array.sty patches and improves
% the standard LaTeX2e array and tabular environments to provide better
% appearance and additional user controls. As the default LaTeX2e table
% generation code is lacking to the point of almost being broken with
% respect to the quality of the end results, all users are strongly
% advised to use an enhanced (at the very least that provided by array.sty)
% set of table tools. array.sty is already installed on most systems. The
% latest version and documentation can be obtained at:
% http://www.ctan.org/tex-archive/macros/latex/required/tools/


%\usepackage{mdwmath}
%\usepackage{mdwtab}
% Also highly recommended is Mark Wooding's extremely powerful MDW tools,
% especially mdwmath.sty and mdwtab.sty which are used to format equations
% and tables, respectively. The MDWtools set is already installed on most
% LaTeX systems. The lastest version and documentation is available at:
% http://www.ctan.org/tex-archive/macros/latex/contrib/mdwtools/


% IEEEtran contains the IEEEeqnarray family of commands that can be used to
% generate multiline equations as well as matrices, tables, etc., of high
% quality.


%\usepackage{eqparbox}
% Also of notable interest is Scott Pakin's eqparbox package for creating
% (automatically sized) equal width boxes - aka "natural width parboxes".
% Available at:
% http://www.ctan.org/tex-archive/macros/latex/contrib/eqparbox/





% *** SUBFIGURE PACKAGES ***
%\ifCLASSOPTIONcompsoc
%\usepackage[tight,normalsize,sf,SF]{subfigure}
%\else
%\usepackage[tight,footnotesize]{subfigure}
%\fi
% subfigure.sty was written by Steven Douglas Cochran. This package makes it
% easy to put subfigures in your figures. e.g., "Figure 1a and 1b". For IEEE
% work, it is a good idea to load it with the tight package option to reduce
% the amount of white space around the subfigures. Computer Society papers
% use a larger font and \sffamily font for their captions, hence the
% additional options needed under compsoc mode. subfigure.sty is already
% installed on most LaTeX systems. The latest version and documentation can
% be obtained at:
% http://www.ctan.org/tex-archive/obsolete/macros/latex/contrib/subfigure/
% subfigure.sty has been superceeded by subfig.sty.


%\ifCLASSOPTIONcompsoc
%  \usepackage[caption=false]{caption}
%  \usepackage[font=normalsize,labelfont=sf,textfont=sf]{subfig}
%\else
%  \usepackage[caption=false]{caption}
%  \usepackage[font=footnotesize]{subfig}
%\fi
% subfig.sty, also written by Steven Douglas Cochran, is the modern
% replacement for subfigure.sty. However, subfig.sty requires and
% automatically loads Axel Sommerfeldt's caption.sty which will override
% IEEEtran.cls handling of captions and this will result in nonIEEE style
% figure/table captions. To prevent this problem, be sure and preload
% caption.sty with its "caption=false" package option. This is will preserve
% IEEEtran.cls handing of captions. Version 1.3 (2005/06/28) and later 
% (recommended due to many improvements over 1.2) of subfig.sty supports
% the caption=false option directly:
%\ifCLASSOPTIONcompsoc
%  \usepackage[caption=false,font=normalsize,labelfont=sf,textfont=sf]{subfig}
%\else
%  \usepackage[caption=false,font=footnotesize]{subfig}
%\fi
%
% The latest version and documentation can be obtained at:
% http://www.ctan.org/tex-archive/macros/latex/contrib/subfig/
% The latest version and documentation of caption.sty can be obtained at:
% http://www.ctan.org/tex-archive/macros/latex/contrib/caption/




% *** FLOAT PACKAGES ***
%
%\usepackage{fixltx2e}
% fixltx2e, the successor to the earlier fix2col.sty, was written by
% Frank Mittelbach and David Carlisle. This package corrects a few problems
% in the LaTeX2e kernel, the most notable of which is that in current
% LaTeX2e releases, the ordering of single and double column floats is not
% guaranteed to be preserved. Thus, an unpatched LaTeX2e can allow a
% single column figure to be placed prior to an earlier double column
% figure. The latest version and documentation can be found at:
% http://www.ctan.org/tex-archive/macros/latex/base/



%\usepackage{stfloats}
% stfloats.sty was written by Sigitas Tolusis. This package gives LaTeX2e
% the ability to do double column floats at the bottom of the page as well
% as the top. (e.g., "\begin{figure*}[!b]" is not normally possible in
% LaTeX2e). It also provides a command:
%\fnbelowfloat
% to enable the placement of footnotes below bottom floats (the standard
% LaTeX2e kernel puts them above bottom floats). This is an invasive package
% which rewrites many portions of the LaTeX2e float routines. It may not work
% with other packages that modify the LaTeX2e float routines. The latest
% version and documentation can be obtained at:
% http://www.ctan.org/tex-archive/macros/latex/contrib/sttools/
% Documentation is contained in the stfloats.sty comments as well as in the
% presfull.pdf file. Do not use the stfloats baselinefloat ability as IEEE
% does not allow \baselineskip to stretch. Authors submitting work to the
% IEEE should note that IEEE rarely uses double column equations and
% that authors should try to avoid such use. Do not be tempted to use the
% cuted.sty or midfloat.sty packages (also by Sigitas Tolusis) as IEEE does
% not format its papers in such ways.




%\ifCLASSOPTIONcaptionsoff
%  \usepackage[nomarkers]{endfloat}
% \let\MYoriglatexcaption\caption
% \renewcommand{\caption}[2][\relax]{\MYoriglatexcaption[#2]{#2}}
%\fi
% endfloat.sty was written by James Darrell McCauley and Jeff Goldberg.
% This package may be useful when used in conjunction with IEEEtran.cls'
% captionsoff option. Some IEEE journals/societies require that submissions
% have lists of figures/tables at the end of the paper and that
% figures/tables without any captions are placed on a page by themselves at
% the end of the document. If needed, the draftcls IEEEtran class option or
% \CLASSINPUTbaselinestretch interface can be used to increase the line
% spacing as well. Be sure and use the nomarkers option of endfloat to
% prevent endfloat from "marking" where the figures would have been placed
% in the text. The two hack lines of code above are a slight modification of
% that suggested by in the endfloat docs (section 8.3.1) to ensure that
% the full captions always appear in the list of figures/tables - even if
% the user used the short optional argument of \caption[]{}.
% IEEE papers do not typically make use of \caption[]'s optional argument,
% so this should not be an issue. A similar trick can be used to disable
% captions of packages such as subfig.sty that lack options to turn off
% the subcaptions:
% For subfig.sty:
% \let\MYorigsubfloat\subfloat
% \renewcommand{\subfloat}[2][\relax]{\MYorigsubfloat[]{#2}}
% For subfigure.sty:
% \let\MYorigsubfigure\subfigure
% \renewcommand{\subfigure}[2][\relax]{\MYorigsubfigure[]{#2}}
% However, the above trick will not work if both optional arguments of
% the \subfloat/subfig command are used. Furthermore, there needs to be a
% description of each subfigure *somewhere* and endfloat does not add
% subfigure captions to its list of figures. Thus, the best approach is to
% avoid the use of subfigure captions (many IEEE journals avoid them anyway)
% and instead reference/explain all the subfigures within the main caption.
% The latest version of endfloat.sty and its documentation can obtained at:
% http://www.ctan.org/tex-archive/macros/latex/contrib/endfloat/
%
% The IEEEtran \ifCLASSOPTIONcaptionsoff conditional can also be used
% later in the document, say, to conditionally put the References on a 
% page by themselves.




% *** PDF, URL AND HYPERLINK PACKAGES ***
%
%\usepackage{url}
% url.sty was written by Donald Arseneau. It provides better support for
% handling and breaking URLs. url.sty is already installed on most LaTeX
% systems. The latest version can be obtained at:
% http://www.ctan.org/tex-archive/macros/latex/contrib/misc/
% Read the url.sty source comments for usage information. Basically,
% \url{my_url_here}.


\parskip 7.5pt


% *** Do not adjust lengths that control margins, column widths, etc. ***
% *** Do not use packages that alter fonts (such as pslatex).         ***
% There should be no need to do such things with IEEEtran.cls V1.6 and later.
% (Unless specifically asked to do so by the journal or conference you plan
% to submit to, of course. )


% correct bad hyphenation here
\hyphenation{op-tical net-works semi-conduc-tor u-san-do u-sa-mos}

\begin{document}
%
% paper title
% can use linebreaks \\ within to get better formatting as desired
\title{M\'etodos Num\'ericos Avanzados\\  Un sistema de comunicaciones }
%
%
% author names and IEEE memberships
% note positions of commas and nonbreaking spaces ( ~ ) LaTeX will not break
% a structure at a ~ so this keeps an author's name from being broken across
% two lines.
% use \thanks{} to gain access to the first footnote area
% a separate \thanks must be used for each paragraph as LaTeX2e's \thanks
% was not built to handle multiple paragraphs
%
%
%\IEEEcompsocitemizethanks is a special \thanks that produces the bulleted
% lists the Computer Society journals use for "first footnote" author
% affiliations. Use \IEEEcompsocthanksitem which works much like \item
% for each affiliation group. When not in compsoc mode,
% \IEEEcompsocitemizethanks becomes like \thanks and
% \IEEEcompsocthanksitem becomes a line break with idention. This
% facilitates dual compilation, although admittedly the differences in the
% desired content of \author between the different types of papers makes a
% one-size-fits-all approach a daunting prospect. For instance, compsoc 
% journal papers have the author affiliations above the "Manuscript
% received ..."  text while in non-compsoc journals this is reversed. Sigh.

\author{Mar\'ia de la Puerta Echeverr\'ia (50009),
       Teresa Fontanella De Santis (52455) y
	Tom\'as Mehdi (51014)
	\\~\IEEEmembership{Grupo 2 - Instituto Tecnol\'ogico de Buenos Aires (ITBA)}
	
%\IEEEcompsocitemizethanks{\IEEEcompsocthanksitem M. Shell is with the Department
%of Electrical and Computer Engineering, Georgia Institute of Technology, Atlanta,
%GA, 30332.\protect\\
% note need leading \protect in front of \\ to get a newline within \thanks as
% \\ is fragile and will error, could use \hfil\break instead.
%E-mail: see http://www.michaelshell.org/contact.html
%\IEEEcompsocthanksitem J. Doe and J. Doe are with Anonymous University.}% <-this % stops a space
%\thanks{Manuscript received April 19, 2005; revised January 11, 2007.}
}

% note the % following the last \IEEEmembership and also \thanks - 
% these prevent an unwanted space from occurring between the last author name
% and the end of the author line. i.e., if you had this:
% 
% \author{....lastname \thanks{...} \thanks{...} }
%                     ^------------^------------^----Do not want these spaces!
%
% a space would be appended to the last name and could cause every name on that
% line to be shifted left slightly. This is one of those "LaTeX things". For
% instance, "\textbf{A} \textbf{B}" will typeset as "A B" not "AB". To get
% "AB" then you have to do: "\textbf{A}\textbf{B}"
% \thanks is no different in this regard, so shield the last } of each \thanks
% that ends a line with a % and do not let a space in before the next \thanks.
% Spaces after \IEEEmembership other than the last one are OK (and needed) as
% you are supposed to have spaces between the names. For what it is worth,
% this is a minor point as most people would not even notice if the said evil
% space somehow managed to creep in.



% The paper headers
%\markboth{Journal of \LaTeX\ Class Files,~Vol.~6, No.~1, January~2007}%
%{Shell \MakeLowercase{\textit{et al.}}: Bare Demo of IEEEtran.cls for Computer Society Journals}
% The only time the second header will appear is for the odd numbered pages
% after the title page when using the twoside option.
% 
% *** Note that you probably will NOT want to include the author's ***
% *** name in the headers of peer review papers.                   ***
% You can use \ifCLASSOPTIONpeerreview for conditional compilation here if
% you desire.



% The publisher's ID mark at the bottom of the page is less important with
% Computer Society journal papers as those publications place the marks
% outside of the main text columns and, therefore, unlike regular IEEE
% journals, the available text space is not reduced by their presence.
% If you want to put a publisher's ID mark on the page you can do it like
% this:
%\IEEEpubid{0000--0000/00\$00.00~\copyright~2007 IEEE}
% or like this to get the Computer Society new two part style.
%\IEEEpubid{\makebox[\columnwidth]{\hfill 0000--0000/00/\$00.00~\copyright~2007 IEEE}%
%\hspace{\columnsep}\makebox[\columnwidth]{Published by the IEEE Computer Society\hfill}}
% Remember, if you use this you must call \IEEEpubidadjcol in the second
% column for its text to clear the IEEEpubid mark (Computer Society jorunal
% papers don't need this extra clearance.)



% use for special paper notices
%\IEEEspecialpapernotice{(Invited Paper)}



% for Computer Society papers, we must declare the abstract and index terms
% PRIOR to the title within the \IEEEcompsoctitleabstractindextext IEEEtran
% command as these need to go into the title area created by \maketitle.
\IEEEcompsoctitleabstractindextext{%
\renewcommand{\abstractname}{Resumen}

\begin{abstract}
%\boldmath


En el presente art\'iculo se describe la estimaci\'on de un canal de un sistema de comunicaci\'on mediante el m\'etodo de cuadrados m\'inimos. El entorno de programaci\'on que se utiliza es Matlab.
\end{abstract}

\renewcommand{\IEEEkeywordsname}{Palabras claves}
\begin{IEEEkeywords}
 Sistema de comunicaci\'on, estimaci\'on de canal, respuesta al impulso, cuadrados m\'inimos, QR.
\end{IEEEkeywords}}
% IEEEtran.cls defaults to using nonbold math in the Abstract.
% This preserves the distinction between vectors and scalars. However,
% if the journal you are submitting to favors bold math in the abstract,
% then you can use LaTeX's standard command \boldmath at the very start
% of the abstract to achieve this. Many IEEE journals frown on math
% in the abstract anyway. In particular, the Computer Society does
% not want either math or citations to appear in the abstract.

% Note that keywords are not normally used for peerreview papers.

% make the title area
\maketitle
% To allow for easy dual compilation without having to reenter the
% abstract/keywords data, the \IEEEcompsoctitleabstractindextext text will
% not be used in maketitle, but will appear (i.e., to be "transported")
% here as \IEEEdisplaynotcompsoctitleabstractindextext when compsoc mode
% is not selected <OR> if conference mode is selected - because compsoc
% conference papers position the abstract like regular (non-compsoc)
% papers do!
\IEEEdisplaynotcompsoctitleabstractindextext
% \IEEEdisplaynotcompsoctitleabstractindextext has no effect when using
% compsoc under a non-conference mode.


% For peer review papers, you can put extra information on the cover
% page as needed:
% \ifCLASSOPTIONpeerreview
% \begin{center} \bfseries EDICS Category: 3-BBND \end{center}
% \fi
%
% For peerreview papers, this IEEEtran command inserts a page break and
% creates the second title. It will be ignored for other modes.
\IEEEpeerreviewmaketitle
\section{Introducci\'on}
% Computer Society journal papers do something a tad strange with the very
% first section heading (almost always called "Introduction"). They place it
% ABOVE the main text! IEEEtran.cls currently does not do this for you.
% However, You can achieve this effect by making LaTeX jump through some
% hoops via something like:
%
%\ifCLASSOPTIONcompsoc
%  \noindent\raisebox{2\baselineskip}[0pt][0pt]%
%  {\parbox{\columnwidth}{\section{Introduction}\label{sec:introduction}%
%  \global\everypar=\everypar}}%
%  \vspace{-1\baselineskip}\vspace{-\parskip}\par
%\else
%  \section{Introduction}\label{sec:introduction}\par
%\fi
%
% Admittedly, this is a hack and may well be fragile, but seems to do the
% trick for me. Note the need to keep any \label that may be used right
% after \section in the above as the hack puts \section within a raised box.



% The very first letter is a 2 line initial drop letter followed
% by the rest of the first word in caps (small caps for compsoc).
% 
% form to use if the first word consists of a single letter:
% \IEEEPARstart{A}{demo} file is ....
% 
% form to use if you need the single drop letter followed by
% normal text (unknown if ever used by IEEE):
% \IEEEPARstart{A}{}demo file is ....
% 
% Some journals put the first two words in caps:
% \IEEEPARstart{T}{his demo} file is ....
% 
% Here we have the typical use of a "T" for an initial drop letter
% and "HIS" in caps to complete the first word.
\IEEEPARstart{U}n sistema de comunicaciones consta, como m\'inimo, de tres componentes: un transmisor (que env\'ia los datos), un receptor (que los recibe) y el medio por donde se transmiten los datos o \textit{canal}. Dado que \'este \'ultimo modifica la informaci\'on transmitida, es necesario que el receptor pueda conocerlo, para poder interpretar los datos de forma adecuada. Como generalmente no se conoce, resulta necesario estimarlo. A eso se le llama \textit{estimaci\'on del canal}. En el presente informe se estima el canal, utilizando el m\'etodo QR para cuadrados m\'inimos.\\
En la siguiente secci\'on, Metodolog\'ia Utilizada, se describe la misma, as\'i como las pruebas realizadas.
En la \'ultima secci\'on se muestran los resultados obtenidos y se extraen conclusiones. \\
% You must have at least 2 lines in the paragraph with the drop letter
% (should never be an issue)

\section{Metodolog\'ia utilizada}
Para la realizaci\'on de este trabajo, se toman en cuenta las siguientes fases:
\subsection{Modelo empleado}
El transmisor env\'ia un dato sk cada T segundos, donde ${s_0}$ es enviado en $t = 0$, ${s_1}$ en $t = T$, ${s_2}$ en $t = 2T$, ...\\
La modificaci\'on de los datos por el canal se presenta por la denominada \textit{respuesta al impulso del canal} $\{{h_{k}\}}_{k=0}^{L}$, donde $L$ es la longitud de la respuesta al impulso. \\
Adem\'as de ser modificados por el canal, los datos son afectados por ruido blanco Gaussiano aditivo ${N_k}\sim cN(0, \sigma)$.\\
Teniendo todo esto en cuenta, cada $T$ segundos el receptor observa: \\
\begin{equation}
r_n = \sum_{k=0}^{L-1} {h_{k}s_{n-k}}+N_n
\end{equation}
Tambi\'en podemos expresar la Ec. (1) en forma matricial como
\begin{equation}
\vec{r} = H\vec{s}+\vec{N}
\end{equation}
donde
\begin{equation}
\vec{r} = \left(
\begin{array}{c}
r_0\\ r_1\\ ...\\r_M
\end{array}
\right)
 , \vec{s}= \left(
\begin{array}{c}
s_0\\ s_1\\ ...\\s_{M-1}
\end{array}
\right)  , 
\vec{N}=\left(
\begin{array}{c}
N_0\\ N_1\\ ...\\N_{M-1}
\end{array}
\right)
\end{equation}

\begin{equation}
H = \left(
\begin{array}{ccccccccc}
h_0 & 0 & 0 & ... & 0 & 0 & ... & 0 & 0\\ h_1 & h_0 &0&...&0&0&...&0&0\\ ...& ...&...&...&0&0&...&0&0   \\ ...\\h_{L-1}&& h_{L-2}&&... \\ ...&...&...&...&...&...&...&... &... \\ 0&0&0&0&0&0&0&h_1 & h_0
\end{array}
\right) \in \mathbb{R}^{M X M}.
\end{equation}

y $M \geq L$. \\ \\
Otra forma equivalente es la siguiente:
\begin{equation}
\vec{r} = S\vec{h}+\vec{N}
\end{equation}
donde $\vec{r}$ y $\vec{N}$ son como antes y

\begin{equation}
\vec{h} = \left(
\begin{array}{c}
h_0\\ h_1\\ ...\\h_L
\end{array}
\right),
\end{equation}

\begin{equation}
S = \left(
\begin{array}{c}
h_0\\ h_1\\ ...\\h_L
\end{array}
\right) \in \mathbb{R}^{M X L}.
\end{equation}

Lo que nos interesa es recuperar correctamente la se\~nal enviada $\vec{s}$ a partir de la se\~nal recibida $\vec{r}$. Esto no ser\'ia tan dif\'icil si se conociesen $L$ y los $\{{h_{k}\}}$. En
efecto, tomando $M = L$ en la Ec. (2), podemos buscar $\vec{s}$ que minimice:

\begin{equation}
\| H\vec{s}-\vec{r}\|_{2}^{2}  .
\end{equation}
\\

El problema es, sin embargo, que en general ni $L$ ni $\{{h_{k}\}}$ son conocidos: estimarlos
corresponde al problema de estimaci\'on del canal.
Si $L$ es conocido, una forma de estimar el canal es mediante la Ec. (5) y el
m\'etodo de cuadrados m\'inimos. En efecto, t\'omese $M > L$ y env\'iese una se\~nal
conocida por el receptor $\vec{s} \in \mathbb{R}^{M} $ (denominada secuencia de entrenamiento).
Luego, se estima el canal planteando el siguiente problema de cuadrados m\'inimos:
encontrar $\vec{h} \in \mathbb{R}^{L} $ que minimice
\begin{equation}
\| S\vec{h}-\vec{r}\|_{2}^{2}  .
\end{equation}

\subsection{Pruebas realizadas}
Para efectuar las pruebas correspondientes, se considera utilizar un $L={1,10,30,50}$ , aplicando un ruido gaussiano $\sigma=1$. Se considera este valor por ser lo suficientemente chico para no distorsionar demasiado los datos a transmitir. Tambi\'en se prueba con diferentes longitudes de la cadena de entrenamiento $M={32,512,1024}$. \\  Para analizar c\'omo se estima en "condiciones \'optimas" (verbigracia: sin ruido, etc). tambi\'en se prueba con $M=512$ y $\sigma=0$ (con los valores de $L$ ya mencionados). 

\section{Resultados y conclusiones}

Se realizaron cien corridas con los m\'etodos de b\'usqueda DFS, BFS, ID, Greedy y Astar para cada uno de los tres tableros con niveles de dificultad baja, media y alta. Se tom\'o como medida de dificultad del tablero el nivel del juego, cuanto mayor es el nivel del juego, mayor es la dificultad del tablero (es importante aclarar que esta dificultad no tiene relaci\'on con la cantidad de movimientos para sacar la ficha azul, sino en como hacer los movimientos para sacarla). Los resultados mostrados son un promedio de esas pruebas. La disposici\'on de las piezas en los tableros puede verse en las im\'agenes anexas.

Todas las pruebas fueron realizadas con un orden de aplicaci\'on de reglas aleatorio. La generaci\'on de n\'umeros aleatorios en todos los casos fue inicializada con la misma semilla.

La primer bater\'ia de pruebas fue realizada sobre el tablero de dificultad baja sin ninguna poda m\'as que la eliminaci\'on de ciclos\\ Con estas primeras pruebas se busca analizar si los resultados obtenidos con los distintos m\'etodos de b\'usqueda se corresponden con la teor\'ia.

Los resultados de las pruebas pueden verse en la secci\'on anexa.

La segunda bater\'ia de pruebas fue realizada con los m\'etodos de poda propios del motor y adem\'as se agreg\'o la eliminaci\'on de nodos repetidos con el mismo costo. Esto se puede realizar ya que en este problema todas las reglas tienen el mismo costo asociado. Por lo tanto dos nodos repetidos con el mismo costo van a estar siempre en el mismo nivel del \'arbol y no va a ocurrir que uno pueda ser soluci\'on \'optima y el otro no.\\
Con estas segundas pruebas se busca analizar c\'omo se comportan los distintos m\'etodos de b\'usqueda para este problema en particular, con las correspondientes mejoras en cuanto a eficiencia, que pueden ser aplicadas debido a las caracter\'isticas del mismo.

Los resultados de las pruebas pueden verse en la secci\'on anexa.

\textit{Aclaraciones: 
La fila "nodos visitados sin reiniciar" es un valor relevante s\'olo para el m\'etodo ID y cuenta la cantidad de nodos visitados, sin reiniciar el contador en cada iteraci\'on.\\
Con algunos m\'etodos en determinadas circunstancias (dificultad del tablero, tipo de heur\'istica, etc.) no se llegaron a completar las cien pruebas debido a tiempo de ejecuci\'on o falta de memoria. En estos casos el promedio se calcul\'o sobre el n\'umero de pruebas alcanzadas.}

%\begin{figure}[H]
%\label{fig:refuerzos2}
%\begin{center}
%\includegraphics[width=3.5in]{../plots/test6.2}
% \DeclareGraphicsExtensions.
%\caption{1 entrada, 3 neuronas capa oculta, 1 salida}
%\end{center}
%\end{figure}




%\begin{figure}[H]
%\label{fig:nonfloat}
%\begin{center}
%\centering
%\includegraphics[width=5.5in]{tan_etas}
% \DeclareGraphicsExtensions.
%\caption{Aprendizaje utilizando la funcion tangente hiperb\'olica, variando el coeficiente de aprendizaje }
%\end{center}
%\end{figure}


% Note that IEEE typically puts floats only at the top, even when this
% results in a large percentage of a column being occupied by floats.
% However, the Computer Society has been known to put floats at the bottom.


% An example of a double column floating figure using two subfigures.
% (The subfig.sty package must be loaded for this to work.)
% The subfigure \label commands are set within each subfloat command, the
% \label for the overall figure must come after \caption.
% \hfil must be used as a separator to get equal spacing.
% The subfigure.sty package works much the same way, except \subfigure is
% used instead of \subfloat.
%
%\begin{figure*}[!t]
%\centerline{\subfloat[Case I]\includegraphics[width=2.5in]{subfigcase1}%
%\label{fig_first_case}}
%\hfil
%\subfloat[Case II]{\includegraphics[width=2.5in]{subfigcase2}%
%\label{fig_second_case}}}
%\caption{Simulation results}
%\label{fig_sim}
%\end{figure*}
%
% Note that often IEEE papers with subfigures do not employ subfigure
% captions (using the optional argument to \subfloat), but instead will
% reference/describe all of them (a), (b), etc., within the main caption.





% Note that IEEE does not put floats in the very first column - or typically
% anywhere on the first page for that matter. Also, in-text middle ("here")
% positioning is not used. Most IEEE journals use top floats exclusively.
% However, Computer Society journals sometimes do use bottom floats - bear
% this in mind when choosing appropriate optional arguments for the
% figure/table environments.
% Note that, LaTeX2e, unlike IEEE journals, places footnotes above bottom
% floats. This can be corrected via the \fnbelowfloat command of the
% stfloats package.



\section{Conclusiones}

De la primer bater\'ia de pruebas se puede ver que los datos obtenidos se corresponden con la teor\'ia.\\
Comparando BFS con ID se ve que ambos llegan a la misma profundidad. Ambos m\'etodos son completos cuando el m\'aximo n\'umero de sucesores de un nodo es finito y \'optimos cuando el costo no decrece con la profundidad. Estas dos cosas suceden en el presente problema con lo cual ambos m\'etodos son completos y \'optimos. \ Sin embargo se puede ver que ID en el nivel final expande una cantidad de nodos menor que BFS. Adem\'as BFS tiene una cantidad mucho mayor de nodos frontera que ID. Esto se debe a que ID va visitando nodos y si no son soluci\'on los expande, siempre y cuando no est\'en en el \'ultimo nivel de la iteraci\'on. Por otro lado BFS expande siempre que visita un nodo que no es soluci\'on independientemente del nivel y es por esto que va acumulando m\'as nodos frontera. Esto hace que gaste mayor cantidad de memoria.\\
Comparando DFS con Greedy se ve que Greedy llega a la soluci\'on en una profundidad menor debido a que es un m\'etodo de b\'usqueda informado.\\
Se puede ver tambi\'en que Astar es el m\'etodo que llego a la soluci\'on \'optima expandiendo la menor cantidad de nodos. Comparando las heur\'isticas en este m\'etodo se observa que la segunda al ser admisible expande menor cantidad de nodos que la primera ya que encuentra antes la soluci\'on.\\

De la segunda bater\'ia de pruebas se puede ver que la poda beneficia a todos los m\'etodos de b\'usqueda. El que m\'as se beneficia es BFS en el que se ve que por ejemplo la cantidad de nodos frontera disminuy\'o considerablemente. Esto se debe a que expande una cantidad de nodos mucho menor porque no visita nodos repetidos.\\
A medida que se aumenta la dificultad del tablero se ve que algunos m\'etodos de b\'usqueda comienzan a fallar en algunas situaciones. Por ejemplo en el tablero de dificultad alta ni DFS ni Greedy pueden completar todas las iteraciones. En este tablero tambi\'en se puede ver que a este nivel de dificultad Astar comienza a ser el m\'etodo m\'as \'optimo en cuanto a cantidad de nodos expandidos y nodos frontera (tambi\'en en cuanto al tiempo de ejecuci\'on a pesar de que no sea un par\'ametro de comparaci\'on tan preciso como el resto) y se separa claramente de los dem\'as. Adem\'as se puede observar que la heur\'istica admisible da resultados mucho mejores que la no admisible.

%De los resultados obtenidos se puede concluir que DFS es el m\'etodo que m\'as r\'apido llega a una soluci\'on pero a su vez debido a sus caracter\'isticas es el que m\'as profundidad alcanza.\\
%El m\'etodo BFS por su forma de recorrer los nodos alcanza un nivel de profundidad mucho menor que DFS. Sin embargo la cantidad de nodos expandidos por BFS es mayor en todos los casos debido a que expande todos los nodos de cada nivel mientras que DFS s\'olo expande una determinada rama.\\
%Tanto BFS y ID llegan a la soluci\'on en la misma profundidad lo cual se corresponde con la teor\'ia porque ID es una alternativa a BFS pero que consume menos memoria. O sea, que  ID es completo cuando el m\'aximo n\'umero de sucesores de un nodo es finito y es \'optimo cuando el costo no decrece con la profundidad pero la memoria consumida es menor que en BFS. Esto es as\'i por el funcionamiento de ID que es como un DFS por niveles. Si analiz\'aramos la cantidad de memoria utilizada por cada uno veer\'iamos que ID usa mucha menos.


%Con el modelo que propusimos de la batalla de Iwo Jima, expresado por las ecuaciones diferenciales 8 y 9, 
%nos aproximamos considerablemente a los valores reales. Sin embargo, el modelo puede ser mejorado agregando a las ecuaciones factores que 
%pueden influir en la cantidad de bajas de cada ej\'ercito, como por ejemplo estimar una tasa de p\'erdidas operativas
%que puede depender de la calidad m\'edica, la alimentaci\'on, o las deserciones de ambas fuerzas.







% if have a single appendix:
%\appendix[Proof of the Zonklar Equations]
% or
%\appendix  % for no appendix heading
% do not use \section anymore after \appendix, only \section*
% is possibly needed

% use appendices with more than one appendix
% then use \section to start each appendix
% you must declare a \section before using any
% \subsection or using \label (\appendices by itself
% starts a section numbered zero.)
%


%\appendices
%\section{Proof of the First Zonklar Equation}
%Appendix one text goes here.

% you can choose not to have a title for an appendix
% if you want by leaving the argument blank
%\section{}
%Appendix two text goes here.


% use section* for acknowledgement
%\ifCLASSOPTIONcompsoc
  % The Computer Society usually uses the plural form
 % \section*{Acknowledgments}
%\else
  % regular IEEE prefers the singular form
%  \section*{Acknowledgment}
%\fi


%The authors would like to thank...


% Can use something like this to put references on a page
% by themselves when using endfloat and the captionsoff option.
%\ifCLASSOPTIONcaptionsoff
  %\newpage
%\fi



% trigger a \newpage just before the given reference
% number - used to balance the columns on the last page
% adjust value as needed - may need to be readjusted if
% the document is modified later
%\IEEEtriggeratref{8}
% The "triggered" command can be changed if desired:
%\IEEEtriggercmd{\enlargethispage{-5in}}

% references section

% can use a bibliography generated by BibTeX as a .bbl file
% BibTeX documentation can be easily obtained at:
% http://www.ctan.org/tex-archive/biblio/bibtex/contrib/doc/
% The IEEEtran BibTeX style support page is at:
% http://www.michaelshell.org/tex/ieeetran/bibtex/
%\bibliographystyle{IEEEtran}
% argument is your BibTeX string definitions and bibliography database(s)
%\bibliography{IEEEabrv,../bib/paper}
%
% <OR> manually copy in the resultant .bbl file
% set second argument of \begin to the number of references
% (used to reserve space for the reference number labels box)


\begin{thebibliography}{1}

\bibitem{Libro}
John Mattews and Kurtis Fink, \emph{M\'etodos Num\'ericos con Matlab}
%\bibitem{Filminas}
%Maria Cristina Parpaglione, \emph{Clase 1-2-3.}
%\bibitem{Link}
%http://www.addictinggames.com/puzzle-games/gridlock.jsp, \emph{niveles del juego}
\end{thebibliography}


\newpage
\clearpage
{\bf\underline{Pruebas sin poda}}

Pruebas tablero dificultad baja

\begin{tabular}{l c c c c c c c c}
\hline\hline
Estrategia & DFS & BFS & ID & Greedy (h1) & Greedy (h2)(*1) & Astar (h1) & Astar (h2)\\
\hline
Profundidad	 & 128 & 4 & 4 & 121 & 69 & 4 & 4\\
Nodos expandidos 	 & 130 & 4091 & 3382 & 121 & 69 & 3289 & 2030\\
Nodos expandidos sin reiniciar & 130 & 4091 & 4355 & 121 & 69 & 3289 & 2030\\
Nodos frontera 	& 815 & 31985 & 22 & 776 & 529 & 25764 & 18017\\
Total estados 	& 947 & 36077 & 3405 & 898 & 599 & 29055 & 20048\\
Tiempo (ms)		& 14 & 369 & 47 & 14 & 7 & 2415 & 1434\\
\hline\hline
\label{board1_sinpoda}
\end{tabular}

\textit{(*1) Promedio de 4 corridas}\\

{\bf\underline{Pruebas con poda}}

Pruebas tablero dificultad baja

\begin{tabular}{l c c c c c c c c}
\hline\hline
Estrategia & DFS & BFS & ID & Greedy (h1) & Greedy (h2) & Astar (h1) & Astar (h2)\\
\hline
Profundidad	 & 66 & 4 & 4 & 66 & 49 & 4 & 4\\
Nodos expandidos  & 125 & 256 & 313 & 97 & 346 & 221& 192\\
Nodos expandidos sin reiniciar & 125 & 256 & 558 & 97 & 346 & 221& 192\\
Nodos frontera 	& 264 & 155 & 12 & 266 & 234 & 172 & 266\\
Total estados 	& 391 & 412 & 327 & 365 & 582 & 394 & 460\\
Tiempo (ms)		& 27 & 38 & 27 & 18 & 144 & 33 & 33\\
\hline\hline
\label{board1_conpoda}
\end{tabular}

Pruebas tablero dificultad media

\begin{tabular}{l c c c c c c c c}
\hline\hline
Estrategia & DFS & BFS & ID & Greedy (h1) & Greedy (h2) & Astar (h1) & Astar (h2)\\
\hline
Profundidad	 & 211 & 14 & 14 & 202 & 174 & 14 & 14\\
Nodos expandidos 	 & 976 & 1215 & 5371 & 727 & 588 & 1188 & 1283\\
Nodos expandidos sin reiniciar & 976 & 1215 & 22528 & 727 & 588 & 1188 & 1283\\
Nodos frontera 	& 702 & 60 & 23 & 671 & 586 & 73 & 179 \\
Total estados 	& 1679 & 1276 & 5395 & 1399 & 1175 & 1262 & 1463\\
Tiempo (ms)		& 698 & 297 & 7161 & 405 & 441 & 297 & 345\\
\hline\hline
\label{board2_conpoda}
\end{tabular}

%\clearpage
Pruebas tablero dificultad alta

\begin{tabular}{l c c c c c c c c}
\hline\hline
Estrategia & DFS(*1) & BFS & ID & Greedy (h1)(*2) & Greedy (h2)(*3) & Astar (h1) & Astar (h2)\\
\hline
Profundidad	 & 293 & 10 & 10 & 105 & 57 & 10 & 10\\
Nodos expandidos 	 & 312 & 1605 & 3360 & 111 & 57 & 1430 & 784\\
Nodos expandidos sin reiniciar & 312 & 1605 & 9596 & 111 & 57 & 1430 & 784\\
Nodos frontera 	& 1249 & 716 & 22 & 389 & 214 & 737 & 539\\
Total estados 	& 1562 & 2323 & 3384 & 501 & 272 & 2169 & 1325\\
Tiempo (ms)		& 90 & 491 & 1847 & 20 & 77 & 426 & 200\\
\hline\hline
\label{board3_conpoda}
\end{tabular}

\textit{(*1) Promedio de 11 corridas}
\par\textit{(*2) Promedio de 7 corridas}
\par\textit{(*3) Una corrida}

\begin{figure*}[hp]
\centering
\includegraphics[scale=0.4]{photo1.jpg}
\caption{Tablero dificultad baja}
\label{board1}
\end{figure*}

\begin{figure*}[hp]
\centering
\includegraphics[scale=0.4]{photo1.jpg}
\caption{Tablero dificultad media}
\label{board2}
\end{figure*}

\begin{figure*}[hp]
\centering
\includegraphics[scale=0.4]{photo1.jpg}
\caption{Tablero dificultad alta}
\label{board3}
\end{figure*}

\end{document}